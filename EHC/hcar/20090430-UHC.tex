\begin{hcarentry}[section,new]{UHC, Utrecht Haskell Compiler (aka EHC, ``Essential Haskell'' compiler)}
\report{Atze Dijkstra}%11/08
\status{active development}
\participants{Jeroen Fokker,
  Doaitse S. Swierstra,
  Arie Middelkoop,
  Luc\'ilia Camar\~ao de Figueiredo,
  Carlos Camar\~ao de Figueiredo}
\label{uhc}
\label{ehc}
\makeheader

\paragraph{What is UHC? and EHC?}
UHC is the Utrecht Haskell Compiler, supporting almost all Haskell98 features plus many
experimental extensions.
The first release of UHC was announced on april 18 at the 5th Haskell Hackathon, held in Utrecht.

EHC is the Essential Haskell Compiler project, a series of compilers of which the last is UHC, plus 
an aspectwise organized infrastructure for facilitating experimentation and extension.

The end-user will probably only be aware of UHC as a Haskell compiler,
whereas compiler writers will be more
aware of the internals known as EHC.
The name EHC however will disappear over time, both EHC and UHC together will be branded as UHC.

UHC in its current state still very much is work in progress.
Although we feel it is stable enough to offer the public,
much work needs to be done to make it usable for serious development work.
By design its strong point is the internal aspectwise organisation which we started as EHC.
UHC also offers more advanced and experimental features like higher-ranked polymorphism, partial type signatures,
and local instances.

\paragraph{Under the hood}

For the description of UHC an Attribute Grammar system (AG) is used as well as other
formalisms allowing compact notation like parser combinators.  For the
description of type rules, and the generation of an AG implementation for
those type rules, we use the Ruler system.
For source code management we use Shuffle, which allows partitioning the system into a sequence of steps and aspects.
(Both Ruler and Shuffle are included in UHC).

The implementation of UHC also tackles other issues:
\begin{itemize}
\item
   To deal with the inherent complexity of a compiler the implementation of UHC is organized as a series of
   increasingly complex steps.
   Each step corresponds to a Haskell subset which itself is an extension
   of the previous step.
   The first step starts with the essentials, namely typed lambda
   calculus; the last step corresponds to UHC.

\item
   Independent of each step the implementation is organized into a set of aspects.
   Currently the type system and code generation are defined as aspects,
   which can then be left out so the remaining part can be used as a barebones starting point.

\item
   Each combination of step + aspects corresponds to an actual, that is, an executable compiler.
   Each of these compilers is a compiler in its own right.

\item
   The description of the compiler uses code fragments which are
   retrieved from the source code of the compilers.
   In this way the description and source code are kept synchronized.
\end{itemize}

Part of the description of the series of EH compilers is available
as a PhD thesis.

\paragraph{What is UHC's status, what is new?}
\begin{itemize}
\item
   UHC has seen daylight as a first release.
   At the time of this writing the current release is 1.0.1, fixing reported installation problems.
\item
   Previously started work is still continuing: GRIN backend, full program analysis (Jeroen Fokker),
   type system formalization and automatic generation from type rules
   (Luc\'ilia Camar\~ao de Figueiredo, Arie Middelkoop).
%   A GRIN (Graph Reduction Intermediate Notation) based backend is available,
%   offering global program optimization and code generation to C (done by Jeroen Fokker) as well as LLVM (done by John van Schie).
%\item
%   Work has started on formalizing UHC's type system; extending our Ruler system will be part of this effort
%   (by Luc\'ilia Camar\~ao de Figueiredo, Carlos Camar\~ao de Figueiredo, Arie Middelkoop, Atze Dijkstra).
%\item
%   The organization of UHC into aspects, allowing better partial reuse of UHC.
%\item
%   Though not a direct part of UHC, its supporting tools (AG, Shuffle) are regularly adapted to allow a cleaner UHC code base.
%\item
%   Arie Middelkoop will continue with the development of the Ruler system~\cref{ruler}. 
\end{itemize}

\paragraph{What will happen with UHC in the near future?}
We plan to do the following:

\begin{itemize}
\item
  Improving installation of UHC and its use as a Haskell compiler: use of Cabal, adding missing Haskell98 features.
\item
  Work on adding static analyses (such as strictness analysis), to enable optimizations.
\end{itemize}

\FurtherReading
\begin{compactitem}
\item UHC Homepage:
\url{http://www.cs.uu.nl/wiki/UHC/WebHome}

\item Attribute grammar system:
\url{http://www.cs.uu.nl/wiki/HUT/AttributeGrammarSystem}

\item Parser combinators:
\url{http://www.cs.uu.nl/wiki/HUT/ParserCombinators}

\item Shuffle:
\url{http://www.cs.uu.nl/wiki/Ehc/Shuffle}

\item Ruler:
\url{http://www.cs.uu.nl/wiki/Ehc/Ruler}
\end{compactitem}
\end{hcarentry}
