% UHCUtrechtHaskellCompiler-AU.tex
\begin{hcarentry}[section,updated]{UHC, Utrecht Haskell Compiler}
\label{uhc}
\label{ehc}
\report{Atze Dijkstra}%11/14
\status{active development}
\participants{many others%Jeroen Fokker,
% Doaitse Swierstra,
% Arie Middelkoop,
% Luc\'ilia Camar\~{a}o de Figueiredo,
% Carlos Camar\~{a}o de Figueiredo,
% Andres L\"oh, Jos\'e~Pedro Magalh\~{a}es,
% Vincent van Oostrum, Clemens Grabmayer, Jan Rochel,
% Tom Lokhorst, Jeroen Leeuwestein, Atze van der Ploeg, Paul van der Ende, Calin Juravle, Levin Fritz
}
\makeheader


UHC is the Utrecht Haskell Compiler, supporting almost all Haskell98 features and most of Haskell2010, plus
experimental extensions.

\paragraph{Status}

Current active development directly on UHC:
\begin{compactitem}
\item Making intermediate Core language available as a compilable language on its own (Atze Dijkstra) to be used for experimenting with alternate Agda backends (Philipp Hausmann, released, talk at upcoming TFP).
\item The platform independent part of UHC has been made available via Hackage, as package ``uhc-light'' together with a small interpreter for Core files (Atze Dijkstra, interpreter still under development).
\item Implementing static analyses (various students, Jurriaan Hage).
\end{compactitem}

Current work indirectly on or related to UHC:
\begin{compactitem}
\item Incrementality of analysis via the Attribute Grammar system used to construct UHC (Jeroen Bransen, PhD thesis finished (soon to be defended), see also UUAGC).
\item Rewriting the type system combining ideas from the constrained-based approach in GHC and type error improvements found in Helium (Alejandro Serrano).
\end{compactitem}

\paragraph{Background.}

UHC actually is a series of compilers of which the last is UHC, plus
infrastructure for facilitating experimentation and extension.
The distinguishing features for dealing with the complexity of the compiler and for experimentation are
(1) its stepwise organisation as a series of increasingly more complex standalone compilers,
the use of DSL and tools for its (2) aspectwise organisation (called Shuffle) and
(3) tree-oriented programming (Attribute Grammars, by way of the
Utrecht University Attribute Grammar (UUAG) system~\cref{uuag}.

\FurtherReading
\begin{compactitem}
\item UHC Homepage:
\url{http://www.cs.uu.nl/wiki/UHC/WebHome}

\item UHC Github repository:
\url{https://github.com/UU-ComputerScience/uhc}

\item Attribute grammar system:
\url{http://www.cs.uu.nl/wiki/HUT/AttributeGrammarSystem}

\end{compactitem}
\end{hcarentry}
