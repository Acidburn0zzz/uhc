% UHCUtrechtHaskellCompiler-AU.tex
\begin{hcarentry}[section,updated]{UHC, Utrecht Haskell Compiler}
\label{uhc}
\label{ehc}
\report{Atze Dijkstra}%05/11
\status{active development}
\participants{many others%Jeroen Fokker,
% Doaitse Swierstra,
% Arie Middelkoop,
% Luc\'ilia Camar\~{a}o de Figueiredo,
% Carlos Camar\~{a}o de Figueiredo,
% Andres L\"oh, Jos\'e~Pedro Magalh\~{a}es,
% Vincent van Oostrum, Clemens Grabmayer, Jan Rochel,
% Tom Lokhorst, Jeroen Leeuwestein, Atze van der Ploeg, Paul van der Ende, Calin Juravle, Levin Fritz
}
\makeheader

\paragraph{What is new?}
UHC is the Utrecht Haskell Compiler, supporting almost all Haskell98 features and most of Haskell2010, plus
experimental extensions.
Since the last release a Javascript backend has been implemented.
We plan to make a next release autumn this year.

\paragraph{What do we currently do and/or has recently been completed?}
As part of the UHC project, the following (student) projects and other activities are underway (in arbitrary order):
\begin{itemize}
\item Jeroen Bransen (PhD): ``Incremental Global Analysis''.
\item Jan Rochel (PhD): ``Realising Optimal Sharing'', based on work by Vincent van Oostrum and Clemens Grabmayer.
\item Arie Middelkoop (PhD, to be defended soon): type system formalization and automatic generation from type rules, in particular the Attribute Grammar variants Ruler-Core for supporting more complex type system implementations.
\item Tamar Christina: an implementation of HML using Ruler-Core.
\item Jeroen Leeuwestein: incrementalization of whole program analysis.
\item Jeroen Fokker: GRIN backend, whole program analysis.
\item Doaitse Swierstra: parser combinator library.
\item Atze Dijkstra: overall architecture, type system, bytecode interpreter + java + javascript backend, garbage collector.
\end{itemize}

\paragraph{Background}

UHC actually is a series of compilers of which the last is UHC, plus
infrastructure for facilitating experimentation and extension.
The distinguishing features for dealing with the complexity of the compiler and for experimentation are
(1) its stepwise organisation as a series of increasingly more complex standalone compilers,
the use of DSL and tools for its (2) aspectwise organisation (called Shuffle) and
(3) tree-oriented programming (Attribute Grammars, by way of the
Utrecht University Attribute Grammar (UUAG) system~\cref{uuag}.
%
%For more information, see the references provided.

\FurtherReading
\begin{compactitem}
\item UHC Homepage:
\url{http://www.cs.uu.nl/wiki/UHC/WebHome}

\item UHC Blog:
\url{http://utrechthaskellcompiler.wordpress.com}

\item Attribute grammar system:
\url{http://www.cs.uu.nl/wiki/HUT/AttributeGrammarSystem}

\item Parser combinators:
\url{http://www.cs.uu.nl/wiki/HUT/ParserCombinators}

\item Shuffle:
\url{http://www.cs.uu.nl/wiki/Ehc/Shuffle}

\item Ruler:
\url{http://www.cs.uu.nl/wiki/Ehc/Ruler}
\end{compactitem}
\end{hcarentry}
