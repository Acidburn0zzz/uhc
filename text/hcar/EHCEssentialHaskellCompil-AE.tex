\begin{hcarentry}{EHC, `Essential Haskell' Compiler}
\label{ehc}
\report{Atze Dijkstra}
\status{active development}
\participants{Atze Dijkstra, Jeroen Fokker, Arie Middelkoop, Doaitse Swierstra}
\entry{unchanged, unpinged}% done, 09.06.2006, remove 11/2006
\makeheader

The purpose of the EHC project is to provide a description of a Haskell
compiler which is as understandable as possible so it can be used for
education as well as research.

For its description an Attribute Grammar system (AG) is used as well as other
formalisms allowing compact notation like parser combinators.  For the
description of type rules, and the generation of an AG implementation for
those type rules, we use the Ruler system~\cref{ruler}
(included in the EHC project).

The EHC project also tackles other issues:
\begin{itemize}
\item
   In order to avoid overwhelming the innocent reader,
   the description of the compiler is organised as a series of
   increasingly complex steps.
   Each step corresponds to a Haskell subset which itself is an extension
   of the previous step.
   The first step starts with the essentials, namely typed lambda
   calculus.

\item
   Each step corresponds to an actual, that is, an executable compiler.
   Each of these compilers is a compiler in its own right so
   experimenting can be done in isolation of additional complexity
   introduced in later steps.

\item
   The description of the compiler uses code fragments which are
   retrieved from the source code of the compilers.
   In this way the description and source code are kept synchronized.
\end{itemize}

Currently EHC already incorporates more advanced features like
higher-ranked polymorphism, partial type signatures, class system,
explicit passing of implicit parameters (i.e. class instances),
extensible records, kind polymorphism.

Part of the description of the series of EH compilers is available
as a PhD thesis,
which incorporates previously published material on the EHC project.

The compiler is used for small student projects as well as larger
experiments such as the incorporation of an Attribute Grammar system.

\textbf{New:}
we are currently working on the following:
\begin{itemize}
\item
   A Haskell98 frontend, supporting most of Haskell98, done by Atze Dijkstra.
\item
   A GRIN (Graph Reduction Intermediate Notation \cite{boquist99phd-optim-lazy}) like backend,
   which allows experimenting with global program optimization.
   This is done by Jeroen Fokker.
\item
   Arie Middelkoop will continue with the development of the Ruler system~\cref{ruler}. 
\end{itemize}

\FurtherReading
\begin{compactitem}
\item Homepage:

\url{http://www.cs.uu.nl/groups/ST/Ehc/WebHome}

\item Attribute grammar system:

\url{http://www.cs.uu.nl/wiki/HUT/AttributeGrammarSystem}

\item Parser combinators:

\url{http://www.cs.uu.nl/wiki/HUT/ParserCombinators}
\end{compactitem}
\end{hcarentry}


%% Additional reference:
%% @book{boquist99phd-optim-lazy , eprint = {papers/boquist99phd-optim-lazy.pdf} , title = {{Code Optimisation Techniques for Lazy Functional Languages, PhD Thesis}} , author = {Boquist, Urban} , publisher = {Chalmers University of Technology} , year = {1999} , url = {http://www.cs.chalmers.se/~boquist/phd/index.html} , howpublished = {\verb|http://www.cs.chalmers.se/~boquist/phd/index.html|}}
