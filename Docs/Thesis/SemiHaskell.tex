\chapter{Extensions}
\label{cap:Extensions}
Up until now the type checker can only type single one sentence expressions. We can't actually type check a full function because of the fact that there is no support for case statements. This means that type checking an entire Haskell module is also out of the question. This chapter outlines the steps taken to implement type checking of entire modules.
\section{Tuples}
Because of the way functions are desugared in EH to case expressions we need support for tuples. UHC implements tuples quite differently then GHC. UHC implements them as a recursive datatype. The reason for this is that it makes it easier to implement extensible records.

Inside the \emph{Expr} datatype there is this definition:
\begin{code}
  con Rec
    recExpr       :  RecExpr
\end{code}

This particular constructor represents a tuple expression. \emph{RecExpr} is defined as:

\begin{figure}[H]
\begin{minipage}[t]{0.5\linewidth}
\begin{code}
data RecExpr
  con Empty
  con Ext
    recExpr     :  RecExpr
    mbNm        :: (Maybe HsName)
    expr        :  Expr
  con Upd
    recExpr     :  RecExpr
    mbNm        :: HsName
    expr        :  Expr
  con Expr
    expr        :  Expr
\end{code}
\end{minipage}
\begin{minipage}[t]{0.5\linewidth}
\begin{code}
itf RecExpr
  visit infer
    inh pre :: Prefix
    inh env :: Gamma
    inh frs :: Int
    inh ast :: RecExpr
    syn pre :: Prefix
    syn sub :: Env
    syn res :: Maybe RowTyExpr
    syn frs :: Int 
\end{code}
\end{minipage}
\label{record}
\caption{Definition of a record along with the interface}
\end{figure}
Figure \ref{record} also shows the interface defined for typechecking tuples. The result is a \emph{RowTyExpr}. In the case of the \emph{Upd} constructor we're going to ignore that since we don't support records.

\paragraph{extending the datasem Expr}
Extending the Expr datatype is quite straight forward, the only code needed is

\begin{code}
    clause Rec
       recExpr.pre = lhs.pre
       recExpr.env = lhs.env
       recExpr.frs = lhs.frs
       
       loc.res = let e = recExpr.res
                 in case e of
                      Nothing -> error "Type inference failed in tuple"
                      Just ty -> TyExpr_AppTop $
                                    TyExpr_App
                                       (TyExpr_Con (mkName "_Rec"))
                                       (TyExpr_Row ty)
       
       (loc.q1, loc.q2) = split (recExpr.pre, domain lhs.pre)
       lhs.res = return $ desugar $ TyScheme_Sugar loc.q2 $ TyScheme_SystemF loc.res
       lhs.sub = recExpr.sub
       lhs.pre = loc.q1
       lhs.frs = recExpr.frs
\end{code}
The type of the tuple is inferred and wrapped as a call to a special constructor "\_Rec" and then generalized. This is all that's needed in \emph{Expr}

\paragraph{Implementing RecExpr}
The code relevant to the implementation of a \emph{RecExpr} is:

\begin{code}
       recExpr.pre = lhs.pre
       recExpr.env = lhs.env
       recExpr.frs = lhs.frs
       
       expr.pre = recExpr.pre
       expr.env = appAll recExpr.sub lhs.env -- update bindings for e.g. \f -> f 1, f 'a')
       expr.frs = recExpr.frs
\end{code}
We first type check the recursive part of the \emph{RecExpr} and apply the resulting substitutions to the $\Gamma$ to update the shared values again. This is important in expressions such as $\\f \rightarrow (f 1, f 'a')$. With the new expression and prefix we can then infer the type of the expression at the current position of the tuple.

\begin{code}    
       (loc.q1, loc.frs, loc.res)
               = let e = recExpr.res
                     t = expr.res
                 in case (e, t) of
                      (Nothing, _) -> error $ "Cannot infer the type of tuple element '" ++ pp loc.mbNm ++ "'"
                      (_, Nothing) -> error $ "Cannot infer the type of the expression '" ++ pp loc.expr ++ "'"
                      (Just rty,Just ty) -> let (typ, frs, _) = renameBound expr.frs ty 
                                                (q,   e     ) = second (\ty'->RowTyExpr_Ext rty loc.mbNm ty') (deep_explode' expr.pre typ)
                                            in (q, frs, e)
\end{code}
\emph{rty} represents the type of the recursive RecExpr and the \emph{ty} is the type of the current expression. If both are not Nothing then the current type is renamed and it's quantifiers are removed and moved to the prefix. Generalization in the \emph{Expr} datasem will fully quantify them again. We do this because the type of $(id, id)$ should be $\forall a. b. (a \rightarrow a, b \rightarrow b)$ and not $(\forall a. a \rightarrow a, \forall a. a \rightarrow a)$.

\begin{code}                              
       lhs.sub = recExpr.sub `munion` expr.sub
       lhs.res = return loc.res
       lhs.frs = loc.frs
       lhs.pre = loc.q1
    clause Empty
       lhs.res = return RowTyExpr_Empty
\end{code}
The resulting type is then simple returned along with the substitutions and the new prefix. 

\section{Case expressions}
UHC desugars a function with multiple binding into a single binding with a case expression inside. This is done in two ways
\paragraph{Single argument functions}
Consider the function \emph{f} defined as:
\begin{code}
f True = False
f x    = x && True
\end{code}
This is desuged by UHC into:
\begin{code}
let f = \x1 -> case x1 of
                 True -> False
                 x -> (&& x True)
\end{code}

\paragraph{multiple argument functions}
However multiple argument functions are desugared somewhat differently, consider the function \emph{q}:
\begin{code}
q True  x = id x
q False x = (+x)
\end{code}
which is desugared into:
\begin{code}
let q = \x1 x2 -> case (x1,x2) of
                    (True,x) -> id x
                    (False,x) -> \$x_0_3_0 -> (+ $x_0_3_0 x)
\end{code}

In the abstract syntax of EH there's a flag to indicate of there are multiple arguments as well. Adding support for tuples means adding a definition for the \emph{Case} clause in the datasem of the \emph{Expr} type. The datatype definition for the constructor \emph{Case} in \emph{Expr} is:
\begin{code}
  con Case
    expr          :  Expr
    alts          :  CaseAlts
    mbCaseIds     :: (Maybe UIDS)
    caseFailS     :: UIDS
    isTupOfArg    :: Bool
\end{code}

With this definition in hand we can extend the datasem of \emph{Expr}:

\begin{code}
    clause Case
       expr.env = lhs.env
       expr.pre = lhs.pre
       expr.frs = lhs.frs
       
       loc.etype = maybe (error $ "Type inference of expression '" ++ pp loc.expr ++ "' failed in Case") id expr.res
\end{code}
We start off by first inferring the type of the scrutinee. This is usually returned just completely polymorphic types. 

\begin{code}
       alts.exp = loc.etype
       alts.env = appAll expr.sub lhs.env
       alts.frs = expr.frs
       alts.pre = expr.pre
       
       child s : UnifyScheme = unifyScheme
       s.pre = alts.pre
       s.ty1 = loc.etype
       s.ty2 = caseLHS alts.res
       s.frs = alts.frs
\end{code}
Afterwards we infer the types of the case alternatives. The result of which is a record which contains the inferred type for the left hand side of the case alternatives (unified together) and the type representing of the right hand side of the case alternatives. The type of the scrutinee is then unified with that of the left-hand sides of the case alternatives.

\begin{code}       
       (loc.q1, loc.q2) = split (s.pre, domain lhs.pre)
       
       lhs.res = return $ desugar $ TyScheme_Sugar loc.q2 $ appAll alts.sub $ body alts.res
       lhs.sub = s.sub `pick` alts.sub `pick` expr.sub
       lhs.pre = loc.q1
       lhs.frs = s.frs
\end{code}
As is the custom, at the end we generalize and return the fully quantified type, which consists of the applications gathered during substitution to the type of the right-hand side of the case expressions (to insure that they have all been applied). 

The interface for \emph{CaseAlts} is defined as:
\begin{code}
itf CaseAlts
  visit infer
    inh ast :: CaseAlts
    inh env :: Gamma
    inh pre :: Prefix
    inh frs :: Int
    syn pre :: Prefix
    syn sub :: Env
    syn res :: CaseType
    syn frs :: Int 
\end{code}

The full implementation won't be covered here, but those interested can look it up in the source code.

\subsection{typing rules expression}
Inferring the type of case expressions comes down two the following typing rules which are self explanatory:

\begin{prooftree}
		\AxiomC{$Q,\Gamma \vdash e_1 : \varphi_1 \quad Q,\Gamma \vdash e_2 : \varphi_2$}
	\LeftLabel{Case:\quad}
		\UnaryInfC{$Q,\Gamma \vdash \textbf{case} \hspace{3pt} e_1 of e_2 : \varphi_2$}
\end{prooftree}

\begin{prooftree}
		\AxiomC{$Q,\Gamma \vdash p_1 : \varphi_1 \rightarrow e_1 : \varphi_2 \quad Q,\Gamma \vdash p_2 : \varphi_3 \rightarrow e_2 : \varphi_4 \newline \quad \varphi_5 \sqsubset \varphi_1 \wedge \varphi_3 \quad \varphi_6 \sqsubset \varphi_2 \wedge \varphi_4$}
	\LeftLabel{CaseAlt:\quad}
		\UnaryInfC{$Q,\Gamma \vdash p : \varphi_5 \rightarrow e : \varphi_6$}
\end{prooftree} 
\section{Fixpoint recursion}
Typing fixpoint recursive functions is a simple yet needed extension. What this essentially comes down to is providing support for \emph{letrec}. As an example the function fix defined as

\begin{code}
fix f = f (fix f)
\end{code} 

would be typed as $\forall \alpha \beta. (\alpha \rightarrow \beta) \rightarrow \beta$ without this support. This is because after giving a temporary type for \emph{fix} based on what was known about \emph{f} at the application site, and later learning more about the value of \emph{f}, the type of \emph{fix} is never updated.

Type checking is essentially building up a list of constraints and then solving these constraints. In this case the constraints are the substitutions that are being build every time extra information about a type becomes available.
This means that we have all the needed bits of information to reconstruct the type we think we should have, and the type we're inferred. In order to provide support for letrec three things need to be done:

\begin{itemize}
\item Apply the substitutions (constraints) gathered during inference to the original type of the binding. Remember that functions are preprocessed and given a polymorphic type variable as a type. This gives us the type we think we should have.
\item Solve the constraints. This is done by unifying the type returned from the inference of the expression and with the type reconstructed in step 1. This fills in any missing information between the types.
\item Apply the substitutions gained from the unification to the inferred type. This then results in the final and correct type for the expression.
\end{itemize}
Since Haskell doesn't have a special keyword for \emph{letrec} the same code is also called for normal functions. In the case of normal (non fixpoint recursive) functions the reconstructed type would be the same as the type that was inferred. Unify would just return an empty substitution.

\section{Type annotations}
Type annotations are nothing special. If the type that's being annotated with a new type has the type of a polymorphic variable then that type is just replaced in the $\Gamma$ with the annotated type. Since there is no bidirectional typing giving a type signature for a function is not really supported. The support for annotation is just enough to allow annotation on variables.
