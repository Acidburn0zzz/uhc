\chapter{Essential Haskell}
The abstract language used by the Utrecht Haskell Compiler (UHC) to represent desugared Haskell is called Essential Haskell or EH for short. EH at it's core resembles $\lambda-calculus$ but with added \textbf{Let} bindings. EH represents a Haskell file but with all the syntactical sugar of the HS file removed, leaving only a core set of basic operations that need to be supported.
\section{Syntax}
UHC Consists of different smaller compilers (often referred to as Variants). For this thesis the variant we will be using is called Variant 8. In terms of \emph{Haskell98} all the needed constructs are there except for \emph{type classes}. This variant is simple enough to be implemented in the alloted time, but complete enough that we can implement a large subset of Haskell functionality using it. Including higher-rank functions.

\subsection{Datatypes}
 The most important constructs of EH8 are given below in terms of \emph{Ruler-Core} datatypes.

\subsubsection{Type expression}

\begin{code}
data TyQu
  con TyForall
    
data TyExpr
  con Con
    nm        :: HsName
  con App
    func      :: TyExpr
    arg       :: TyExpr
  con AppTop
    tyExpr    :: TyExpr
  con Parens
    tyExpr    :: TyExpr
  con Ann
    ann       :: TyExprAnn
    tyExpr    :: TyExpr
  con Wild
  con Mono
  con Var
    nm        :: HsName
  con VarWild
    mm        :: HsName
  con Quant
    qu        :: TyQu
    tyVar     :: HsName
    tyExpr    :: TyExpr
  con Forall
    qu        :: TyQu
    tyVar     :: [HsName]
    tyExpr    :: TyExpr    
  con Row
    rowTyExpr :: RowTyExpr
    
data TyVar
  con Var
    nm      :: HsName
    
type TyVars : [TyVar]
\end{code}

Most of these are pretty standard with the exception of \emph{Row} and \emph{AppTop}. The \emph{Row} constructor is used to represent the type of a Tuple, this will be explained later in Chapter ??. 
The \emph{AppTop} constructor is used purely for pretty printing purposes, in particular when pretty printing Constructor applications. The $(\rightarrow)$ constructor is not treated specially in this version of EH, which means $\alpha \rightarrow \beta$ is encoded as $((\rightarrow \alpha) \beta)$. The \emph{App} found under an \emph{AppTop} is pretty printed differently to show the expected type above.

The \emph{Forall} constructor is not part of the defined abstract syntax of the defined EH8 but is there to facilitate pattern matching. It's syntactical sugar turning expressions such as $\forall a. \forall b. a \rightarrow b$ into $\forall a b. a \rightarrow b$. This makes it easier to in one pattern matching get all the variables the bounded variables.

\subsubsection{Expressions \& Patterns \& Declarations}

\begin{code}
data Expr
  con IConst
    int           :: Int
  con CConst
    char          :: Char
  con Con
    nm            :: HsName
  con Var
    nm            :: HsName
  con App
    func          :  Expr
    arg           :  Expr
  con Let
    isStrict      :: Bool
    decls         :  Decls
    body          :  Expr
  con Lam
    arg           :: PatExpr
    body          :  Expr
  con AppTop
    expr          :  Expr
  con Parens
    expr          :  Expr
  con TypeAs
    tyExpr        :: TyExpr
    expr          :  Expr
  con Ann
    ann           :: ExprAnn
    expr          :  Expr
  con AppImpred
    func          :  Expr
    arg           :  Expr
  con SConst 
    str           :: String
  con Case
    expr          :  Expr
    alts          :: CaseAlts
    mbCaseIds     :: (Maybe UIDS)
    caseFailS     :: UIDS
    isTupOfArg    :: Bool
  con DataFields
    dataFieldExpr :: DataFieldExpr
  con Rec
    recExpr       :: RecExpr
  con Sel
    expr          :  Expr
    lbl           :: HsName
  con Undefined
  con CaseAltFail 
    caseId        :: UID
    
data PatExpr
  con IConst
    int              :: Int
  con CConst
    char             :: Char
  con Con
    nm               :: HsName
  con Var
    nm               :: HsName
  con VarAs
    nm               :: HsName
    patExpr          :: PatExpr
  con App
    func             :: PatExpr
    arg              :: PatExpr
  con AppTop
    patExpr          :: PatExpr
  con Parens
    patExpr          :: PatExpr
  con Ann
    ann              :: PatExprAnn
    patExpr          :: PatExpr
  con TypeAs
    tyExpr           :: TyExpr
    patExpr          :: PatExpr
  con SConst
    str              :: String
  con Rec
    recPatExpr       :: RecPatExpr
  con DataFields
    dataFieldPatExpr :: DataFieldPatExpr
  con Irrefutable
    patExpr          :: PatExpr

data DataConstr
  con Constr
    conNm      :: HsName
    fields     :  DataFields
    
type DataConstrs : [DataConstr]

data DataField
  con Field
    mbLabels     :: (Maybe [HsName])
    tyExpr       :: TyExpr
    
type DataFields : [DataField]

data Decl
  con TySig
    nm         :: HsName
    tyExpr     :: TyExpr
  con Val
    patExpr    :: PatExpr
    expr       :  Expr
  con Data
    isNewType  :: Bool
    tyNm       :: HsName
    tyVars     :: TyVars
    constrs    :: DataConstrs
  con KiSig
    nm         :: HsName
    kiExpr     :: KiExpr
  con FFI
    callconv   :: FFIWay
    safety     :: String
    impEnt     :: String
    nm         :: HsName
    tyExpr     :: TyExpr
  
type TyExprs : [TyExpr]
type Decls   : [Decl]
\end{code}

The attentive reader would notice that we have defined datatypes but have not defined any structures for \emph{case} statements. This was done intentionally as \emph{case} statements are an extension of the base algorithm, which will be covered in chapter ??. This version of EH also only has build-in support for the \emph{Int} and \emph{Char} datatypes.

\section{Ordering}
Whenever a Haskell file is desugared into a EH file, some reordering is done. In the EH file function,data declarations and type synonyms are always displayed before any function or other entities that uses them. Or alternatively, for every entity (where entity can be anything that refers to another type/function) a node is created and a dependency graph is made, which is then \emph{Topologically sorted}, that is, for every pair of edge \emph{uv} (\emph{u} is used in \emph{v}) \emph{u} comes before \emph{v}. It's worth noting that datatype and type synonym declarations are always put at the top of the file, before the function declarations.

The first obvious question is, what if functions or datatypes have a cyclic dependency. E.g. mutually recursion:

\begin{code}
data Foo = Foo Bar
data Bar = Bar Foo | NoBar

foo = bar
bar = foo
\end{code}

Aside from the fact that the functions foo and bar never terminate, they should however typecheck perfectly fine. For datatypes nothing special needs to be done to make EH support mutual recursion. When processing datatypes we just always add information about the data \emph{Type} before processing the constructors. For binding processing we do the same. The EH syntax for \emph{Let} allows us to bind multiple \emph{Decl} at once. This means, that We will have one \emph{Let} binding both \emph{foo} and \emph{bar} at the same time.

Any Haskell after desugared to EH is turned into one big \emph{Let} statement terminating in either 0 or \emph{main} depending on if an actual main function is defined in the module.

Figure ?? which does not define a main is desugered into:

\begin{code}
let main = 0
in let data Bar  = Bar Foo | NoBar
       data Foo  = Foo Bar
   in let foo = bar
          bar = foo
      in 0
\end{code}

in this case, since \emph{main} is not dependent on anything it is the top level declaration, in every other cases where main is defined it will be put somewhere inside the dependency graph and the top level \emph{Let} will terminate \emph{in} main.
\section{Semantics}

This ordering that EH brings to a Haskell module is quite handy, especially for Attribute Grammar code. Attribute Grammar compilers do computations on attributes defined on the nodes of a tree. A Haskell file desugared into an EH file is now a Top-down tree, where the only case that needs some special handling is mutual recursion. 

This makes certain tasks a lot easier, for instance when looking up a value in the environment, if we can't find it, then it's really not defined. there's no need to look any further. If we can't type check a specific node, it also means we don't have to bother trying any of it's children, since they depend on the node that failed type checking.
