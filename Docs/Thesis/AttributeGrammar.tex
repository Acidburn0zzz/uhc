\chapter{Attribute Grammars}
\label{AG}
\emph{Context Free Grammars}(CFG) cannot describe the complete syntax of programming languages\cite{knuth1}, particularly specifying any context-sensitive condition is impossible. For example the condition that the same value for \emph{n} be enforced in the string $a^nb^nc^n$ cannot be tested by using context free grammars\cite{ken}.

Developed in 1968 by Donald E. Knuth \ags were created as a way to define \emph{meaning} to context free languages. An example of assigning meaning to a grammar is defining the evaluation of expressions. The expression grammar is defined by the following \emph{CFG} expressed as \emph{BNF}:

\begin{figure}[H]
\begin{grammar}
      [(colon){$\rightarrow$}]
      [(semicolon)$|$]
      [(comma){}]
      [(period){\\}]
      [(quote){\begin{bf}}{\end{bf}}]
      [(nonterminal){$\langle$}{$\rangle$}]
<Expr>   \hspace{52pt} : <number>; <Expr>, <operator>, <Expr>.
<number> \hspace{42pt} : <digit>;<digit>,<number>.
<digit> \hspace{55pt} : "0";"1";"2";"3";"4";"5";"6";"7";"8";"9".
<relational operator>: "$-$";"$+$".
\end{grammar}
\caption{BNF definition for Expressions}
\label{grammar:bnf:expr}
\end{figure}

The \emph{terminals} in this case are the \emph{operators} $+,-$ and the \emph{digits} $0\ldots 9$. The \emph{nonterminals} are the \emph{Expr} and \emph{number} symbols. The grammar specifies that an expression is either a \textbf{number} or an \textbf{Expr} followed by an \textbf{operator} and then another \textbf{Expr}. For every sentence that can be produced by this grammar a parse tree can be assigned. For instance the expression "2 + (3 - 5) + (6 - 1)" produces the parse tree in figure \ref{fig.example1.parsetree}. Meaning is defined for the expression on a step by step basis corresponding with the structure of the parse tree, starting from the leaves and working up the tree. By assigning \emph{attributes} to the tree elements during traversals, meaning can be incrementally defined to the parse tree. See figure \ref{fig.example1.parsetree}

\begin{figure}[H]
\centering
\begin{tikzpicture}[>=stealth']
\node (r0) [draw, circle] {+};
\node (r1) [below left=1cm of r0, draw, circle] {2}
  edge[<-] (r0.south);
\node (r2) [below right=1cm of r0, draw, circle] {+}
  edge[<-] (r0.south);
\node (s0) [below left=1.4cm of r2, draw, circle] {-}
  edge[<-] (r2.south);
\node (s1) [below left=1cm of s0, draw, circle] {3}
  edge[<-] (s0.south);
\node (s2) [below right=1cm of s0, draw, circle] {5}
  edge[<-] (s0.south);
\node (f0) [below right=1.5cm of r2, draw, circle] {-}
  edge[<-] (r2.south);
\node (f1) [below left=1cm of f0, draw, circle] {6}
  edge[<-] (f0.south);
\node (f2) [below right=1cm of f0, draw, circle] {1}
  edge[<-] (f0.south);
\end{tikzpicture}
\caption{Parse tree example for "2 + (3 - 5) + (6 - 1)"}
\label{fig.example1.parsetree}
\end{figure}

AG's are additions to CFGs that propagate semantic information along through parse trees. As with the majority of trees used in computer science, ASTs are created with the \emph{root} at the top and the \emph{leaves} at the bottom. With this in mind there are two kinds of \emph{attributes} defined by knuth\cite{knuth1}:
\begin{description}
\item[\textbf{synthesized}] An attribute that is only dependent on \emph{descendants} of the nonterminal. They are passed bottom-up through the tree.
\item[\textbf{inherited}] An attribute that is defined in terms of the \emph{ancestors} of the nonterminal. They are passed top-down through the tree.
\end{description}

An attribute can be both \emph{synthesized} and \emph{inherited}. Semantics can be defined for the tree in figure \ref{fig.example1.parsetree} by assigning a synthesized attribute named \emph{value} of type \textbf{Int} to the nonterminals \emph{number} and \emph{Expr}. The evaluation rules are:

\begin{figure}[H]
\begin{grammar}
      [(colon){$\rightarrow$}]
      [(semicolon)$||$]
      [(comma){}]
      [(period){\\}]
      [(quote){\begin{bf}}{\end{bf}}]
      [(nonterminal){$\langle$}{$\rangle$}] 
value(<$Expr^+$>):value(<$Expr_1$>) + value(<$Expr_2$>). 
value(<$Expr^-$>):value(<$Expr_1$>) - value(<$Expr_2$>). 
value(<Expr>):value(<number>).
value(<number>):<number>.
\end{grammar}
\caption{attribute definition for Expressions}
\label{semantics:bnf:expr}
\end{figure}

Subscripts are used to disambiguate between the different expression types and  superscripts are used to distinguish between the different cases of the \emph{operator} in an expression. Figure \ref{fig.example2.decoratedtree} shows a tree decorated with the synthesized attribute \emph{v} (short for value) along with intermediate values of \emph{v}.

\begin{figure}[H]
\centering
\begin{tikzpicture}[>=stealth']
\node (r0) [draw, circle] {+};
\node (ri0) [draw, rectangle, right=0.1cm of r0, gray] {v=5};
\node (r1) [below left=1.7cm of r0, draw, circle] {2}
  edge[<-] (r0.south);
\node (ri1) [draw, rectangle, right=0.1cm of r1, gray] {v=3};
\node (r2) [below right=1.7cm of r0, draw, circle] {+}
  edge[<-] (r0.south);
\node (ri2) [draw, rectangle, right=0.1cm of r2, gray] {v=-2};
\node (s0) [below left=2.1cm of r2, draw, circle] {-}
  edge[<-] (r2.south);
\node (ri3) [draw, rectangle, right=0.1cm of s0, gray] {v=15};
\node (s1) [below left=1.0cm of s0, draw, circle] {3}
  edge[<-] (s0.south);
\node (ri4) [draw, rectangle, right=0.1cm of s1, gray] {v=3};
\node (s2) [below right=1.0cm of s0, draw, circle] {5}
  edge[<-] (s0.south);
\node (ri5) [draw, rectangle, right=0.1cm of s2, gray] {v=5};
\node (f0) [below right=2.2cm of r2, draw, circle] {-}
  edge[<-] (r2.south);
\node (ri6) [draw, rectangle, right=0.1cm of f0, gray] {v=5};
\node (f1) [below left=1.0cm of f0, draw, circle] {6}
  edge[<-] (f0.south);
\node (ri7) [draw, rectangle, right=0.1cm of f1, gray] {v=6};
\node (f2) [below right=1.0cm of f0, draw, circle] {1}
  edge[<-] (f0.south);
\node (ri8) [draw, rectangle, right=0.1cm of f2, gray] {v=1};
\end{tikzpicture}
\caption{decorated tree example for "2 + (3 - 5) + (6 - 1)"}
\label{fig.example2.decoratedtree}
\end{figure}

\Ags are akin to \emph{catamorphisms}, except without the need to define the algebra and explicit traversals of the tree.

\section{Utrecht University Attribute Grammar Compiler (UUAGC)}
UUAGC (Utrecht University Attribute Grammar Compiler) is a preprocessor which parses Attribute Grammars in a custom language.
It defines ways to specify an AST, attributes for the terminals and nonterminals and the semantic functions. It is used in the Utrecht Haskell Compiler to do most of the work.

\subsection{Limits of UUAGC}
The \emph{UUAGC} system it limited in that it cannot perform case distinctions over multiple \emph{abstract syntax trees} at the same time\cite{visitag}. For most applications this is not an issue, since in most cases there is only one \emph{AST} that needs to be traversed at a time. 

Unification is the act of trying to find structural \& semantic equivalence between two types. It is a critical part of type checking. Given two types \emph{$t_{1}$} and \emph{$t_{2}$}, unification attempts to find a list of substitutions that allows the instantiation of type \emph{$t_{1}$} to type \emph{$t_{2}$}. For this to be accomplished it needs to be possible to traverse both \emph{$t_{1}$} and \emph{$t_{2}$} concurrently; while comparing the types at every node.

%\begin{figure}[!h]
%\begin{center}
%\begin{neatopic}[width=.5\textwidth, height=50mm]
%    subgraph type1 {
%	  node [];
%	  a1 [label="a"];
%	  a2 [label="b"];
%	  a1 -- a2;
%	  label = "Type 1";
%    }
%
%    subgraph type1 {
%	  node [];
%	  b1 [label="Int"];
%	  b2 [label="Int"];
%        b1 -- b2;
%	  label = "Type 2";
%    }
%    
%    {rank=same; a1 b1}
%    {rank=same; a2 b2}
%
%    a1 -- b1 [style=dotted, label ="a:= Int"];
%    a2 -- b2 [style=dotted, label ="b:= Int"];
%\end{neatopic}
%\end{center}
%\caption{Unification example}
%\label{unify-simple}
%\end{figure}

When unifying the type \emph{$a \rightarrow b$} with \emph{$Int \rightarrow Int$} both trees are traversed concurrently while the nodes are compared, and ultimately terminating with the substitution list [(a,Int), (b, Int)]. The ability to be able to traverse \emph{AST}s that were just produced is also required because the structure of the \emph{type} being produced is not known beforehand. During type inference more type information is gradually gained on the type that needs to be produced. This generally presents a problem for AGs\cite{ruler-front} because the synthesis and evaluation phases are two separate phases.

\section{Ruler-Core}
\Rcore addresses these restrictions in AG by introducing a model based on \emph{visit}\cite{visits} functions. The resulting language is more flexible while still retaining the core semantics of reasoning over decorated trees with attributes. The simplest description of ruler-core would be:\emph{a language to manipulate visit sequences.}

\begin{quotation}
A \emph{visit function}\cite{visitag} is a (partial) function that takes several inputs (\emph{inherited attributes}) and produces several output values (\emph{synthesized attributes}).
\end{quotation}

As with traditional AGs \emph{inherited}, attributes are passed top-down in the tree while \emph{synthesized} attributes are passed bottom-up. An attribute can be both \emph{synthesized} and \emph{inherited}.

Everything that can be expressed in uuagc can be expressed in ruler-core, but the inverse is not true. One of the simplest things that can be done in ruler-core is the evaluation of a single AST. In order to write an evaluator for the expression type in figure \ref{grammar:bnf:expr} we first need to define the \emph{datatypes} and \emph{interfaces}.

\begin{figure}[H]
\begin{minipage}[t]{0.4\linewidth}
\begin{code}
data Expr
  con Num
    val     :: Int
  con Expr
    exp1    :  Expr
    op      :: Operator
    exp2    :  Expr
\end{code}
\end{minipage}
\begin{minipage}[t]{0.6\linewidth}
\begin{code}
data Operator
  con Plus
  con Minus

itf Expr
  visit eval
    inh ast  :: Expr -- input
    syn v    :: Int  -- output
\end{code}
\end{minipage}
\caption{Evaluating expressions in ruler-core: datatypes}
\label{example:tutorial1:datatypes}
\end{figure}

\subsubsection{Data types}
\Rcore data types resemble Haskell's record syntax with some notable differences.

Instead of an $=$ or \textbar \space like in Haskell, an explicit \textbf{con} keyword is used to indicate the \emph{name} of the constructor. Every element of the constructor must be explicitly named. Indentation is also important since indentation separates constructors, in general \textbf{con} needs to be deeper indented then \textbf{data} and the members of a constructor should be indented further than the \textbf{con}. 

Figure \ref{example:tutorial1:datatypes} illustrates two different ways of declaring a type of constructor argument. $:$ is used to declare a type that is a \emph{nonterminal} and $::$ indicates a \emph{terminal}. The reason for this distinction is that for \emph{nonterminal} types some extra machinery is automatically defined. It is important to know that for every nonterminal \rcore enforces that at least one of the declared visits have an \emph{inherited} attribute named \emph{ast} with the type of the nonterminal.

Only types of kind $:: \star$ are allowed by \rcore, which means only monomorphic types are accepted. As is the case with UUAGC, the constructors generated will be in the form of \emph{TypeName\_ConstructorName}. To put this concretely figure \ref{example:tutorial1:datatypes} exposes for the type \emph{Expr} the constructor functions \textbf{Expr\_Num} and \textbf{Expr\_Expr}.

\subsection{Interfaces}
\emph{Interfaces} are analogous to the interface definitions in other languages, except instead of declaring function prototypes/signatures we declare visits and their attributes. Interfaces declare \emph{Non-Terminal}s which can be named the same as their corresponding \emph{data types}. For those familiar with uuagc, a \textbf{ATTR} declaration in uuagc would equal an interface declaration with one visit and all the attributes declared in the \textbf{ATTR} would be part of this one visit. The ability to explicitly declare these visits and interfaces is where ruler-core's true abilities come in.

\begin{figure}[h!]
\begin{code}
itf <name>
  {visit <name>
    {attributes}
  }
\end{code}
\caption{Ruler-core interface declaration syntax}
\label{itf:syntax}
\end{figure}

As many visits as needed can be declared inside an interface. Every visit is a new \emph{co-routine} and will be scheduled to run by ruler-core. Visits are usually executed as soon as possible and as many visits as possible are executed at the same time. % Outside of visits we can also declare attributes. These attributes are thus not explicitly assigned to a visit. They will be automatically assigned to the earliest visit possible.

\subsection{Visits}
Visits are co-routines (functions) that can be invoked to perform a computation, the synthesis of attributes are done in these sequential passes. Visits consist of different clauses. If a visit only has one clause, it does not have to be declared. 

Visits like all functions have arguments, or in this case attributes. The \emph{inh} keyword indicates an \emph{inherited} attribute (input value), whereas \emph{syn} indicates a \emph{synthesized} attribute (output value). The order of declaration of the attributes is not important.

After declaring the datatypes and interfaces; the actual semantic function can be declared to evaluate the expressions:

\begin{figure}[H]
\begin{code}
datasem Expr
   clause Num
     lhs.v = loc.val -- output
   clause Expr
     internal opcheck
       clause Plus
         match Operator.Plus@loc = loc.op
         lhs.v = exp1.v + exp2.v -- output
       clause Minus
         match Operator.Minus@loc = loc.op
         lhs.v = exp1.v - exp2.v -- output
\end{code}
\caption{Evaluating expressions in ruler-core: datasem}
\label{example:tutorial1:datasem}
\end{figure}

Figure \ref{example:tutorial1:datasem} has various elements that can be best explained in isolation. Keep this figure in mind when reading.

\subsection{Semantic functions}
\label{semantics}
Semantic functions are used to define semantics for interfaces. Within a semantic function it is possible to \emph{invoke} any other coroutine(s) that might be needed. Although there is an implicit \textbf{invoke} keyword, it is rarely needed to explicitly \emph{invoke} a visit. When all attributes are defined for a visit it is implicitly invoked.

Within a semantic function it is possible to have any number of semantic rules. Clauses provide a way to do scoping inside these semantic functions. A clause inherits all the attributes of its parent clauses in the same visit. If a visit has only one clause it doesn't have to be explicitly declared. 

There are two ways of defining a semantic function: using the \textbf{datasem} and a \textbf{sem} keyword. The example in figure \ref{example:tutorial1:datasem} uses the former.

\subsubsection{Datasem}
Defining semantics for a \emph{data type} and \emph{nonterminal}(interface) defined in \rcore can be done with a shorthand: \textbf{datasem}. As the name suggests \textbf{datasem} stands for \emph{datatype semantics}. Those familiar with uuagc can compare defining a \textbf{datasem} in \rcore with a \textbf{SEM} declaration in uuagc.

\begin{figure}[!h]
\begin{code}
datasem <nonterminal> [monad <type>]
    {clause <name>
        ...
    }
\end{code}
\caption{Syntax definition of a sem function}
\label{datasem:syntax}
\end{figure}

Defining a \textbf{datasem} is a shorthand for defining a \textbf{sem} (more on this later). The \emph{monad} type does not need to be specified, however if a type was specified for any semantic function which is used by, or used in this \textbf{datasem} then to disambiguate you need to define the type in this declaration as well.

The \emph{clauses} in a \textbf{datasem} should coincide with the constructors of the data type. The preprocessor enforces that there is a clause for every declared constructor. Every clause declaration implicitly adds a \textbf{match} statement for every clause. This is the reason why there is a required attribute \emph{ast} for ever nonterminal. This is the attribute on which matches are tried out on in the main clauses of a \textbf{datasem}. A \textbf{match} is essentially an assertion, if failed nothing else for that clause is tried out and backtracking takes place.

For every \textbf{datasem}, in every clause where there is a nonterminal in the definition of that clause, there will be an implicit child declared for that field. It is for this reason that in figure \ref{example:tutorial1:datasem} it is refered to the operator terminal via the local child (loc.op) and the \emph{|exp1|} and \emph{|exp2|} nonterminals directly. In the case of \emph{Expr} there was only one \emph{inherited} attribute: ast, but since ast is filled in automatically by \rcore the invoke is implicitly performed. Which is why the synthesized attributes can be accessed without any further action. 

Because of all these properties, a \textbf{datasem} only provides the ability to traverse one tree at a time.
\subsection{Bindings}
\label{bindings}
In Haskell the \emph{Let} binding is used when introducing new variables in a sequential computation. In \rcore the keyword \textbf{let} is not used when assigning values to bindings, however since bindings in \rcore are translated to \textbf{let} declarations by the preprocessor the same behaviors is to expected. This means that binding to a \emph{pattern} on the \emph{left hand side} is valid. e.g. \[ (\alpha, \beta) = \ldots \] is allowed. This allows the definition multiple attributes at the same time.

While it is possible to read any attribute as many times as needed, assignment of values to a visits attributes are only allowed once per visit. The compiler will generate an error if it finds code that tries to redefine an attribute (there is no shadowing).

The notation for referencing patterns, expressions and variables is \emph{k}.\emph{x} where \emph{k} is the name of child name and \emph{x} the attribute to be referenced. There are two build in reserved children:

\paragraph{lhs}
The \emph{lhs} child is used to access the \emph{inherited} attributes and to assign values to the \emph{synthesized} attributes. Which one is intended is derived from the context in which they are used: When used at the \emph{left hand side} of an expression they are treated as \emph{synthesized} attributes, but when used in the \emph{right hand side} of a binding they refer to the \emph{inherited} values.
 
\paragraph{loc}
The \emph{loc} child is used in a way that is analogous to local variables in other languages. You can define as many of these as needed. The scope of the \emph{loc} is only the clause/visit that it is declared in and the its children.

\subsection{Clauses}
Clauses are a way of defining alternatives. When an assertion in clause fails  it backtracks out of the clause and the next one is tried out. Clauses are tried out in sequential order. A visit can contain multiple clauses, corresponding to the different ways to interpret the \emph{inherited} attributes of the visit.

If no clause can be executed in a visit, the system backtracks to the parent visit and clause. This behavior goes all the way up to the root. In order to be able to generate proper errors it is recommended to always make the collection of clauses for a visit total. The easiest way to do this is to make a \emph{catch-all} clause at the end.

\subsection{Matches}
Matches are akin to case expressions, like case expressions they force evaluation and attempt to pattern match on the datatype. Except unlike case expressions, you can only on a pattern that contains a constructor. For any data type defined in \rcore itself and the \emph{Bool} type a \textbf{match} can be done.

\begin{figure}[h!]
\begin{code}
match TypeName.ConstructorName@child  = <expression>
\end{code}
\caption{Match syntax definition}
\label{match:syntax}
\end{figure}

If the \emph{match} succeeds the \emph{named} attributes defined for the elements of the Constructor are added as attributes of the specified \emph{child}. Of the two reserved children \textbf{lhs} and \textbf{loc} only \textbf{lhs} is not allowed as a child name here\footnote{note that the childname "var" is also reserved, but in the case of var you will get an actual syntax error}. On the other hand if the \emph{match} fails the entire clause is aborted and backtracking is performed.
There are exceptions to this syntax, \emph{build in} types such as \emph{Bool} which have no children have an alternative syntax:

\begin{figure}[h!]
\begin{code}
match True = <expression>
\end{code}
\caption{Example match on Bools}
\label{match:bool}
\end{figure}

Figure \ref{match:bool} shows that on certain types the rules are relaxed, particularly the Bool type. Note that because there is no build in support for the \emph{Maybe} type, often in this thesis this will be supported by first match on |True = isJust expr| and then a subsequent call the |fromJust| function.

\subsection{Internal}
Sometimes it is necessary to make a decision inside a clause to branch. For example on a value that was just synthesized. This is achieved by using the \emph{internal} keyword. The \emph{internal} keyword provides a means of scoping and branching at the same time. Internals contain a list of \emph{clauses} which will be tried out in order one at a time. Attributes that were declared before the \emph{internal} statement are all in scope inside an internal block. 

\begin{figure}[h!]
\begin{code}
internal <name>
  {clause}
  {clause}
  ...
\end{code}
\caption{Internal syntax definition}
\label{internal:syntax}
\end{figure}

Unfortunately once branched the execution flow never returns back to the parent clause in question. Any code below the internal is floated upward if possible. Referencing a attribute defined in an internal from a parent clause is invalid.

\subsection{Invoking semantic functions}
To complete this example we also need be able to call semantic functions from Haskell:

\begin{figure}[H]
\begin{code}
eval :: Expr -> IO Int
eval exp = do
  let inh = Inh_Expr_eval { ast_Inh_Expr = exp }
  syn <- invoke_Expr_eval dnt_Expr inh
  let x = v_Syn_Expr syn
  return x
\end{code}
\caption{calling wrappers from within haskell}
\end{figure}

The first line (the let) defines the \emph{inherited} attributes expected for the visit that is to be called. In this case the "eval" visit, which specified that there is one \emph{inherited} attribute called \emph{ast}. For every visit there is a record for the \emph{inherited} and \emph{synthesized} attributes. The record name is build up as \textbf{X\_I\_v} where \textbf{X} equals "Inh" or "Syn", \textbf{I} is the interface name and \textbf{v} the visit name.

The labels of the fields inside these record are made up of \textbf{attr\_X\_I} where \textbf{attr} is the attribute name, \textbf{X} either "Inh" or "Syn" indicating the attribute type and \textbf{I} the interface name.

The second line invokes the \textbf{eval} routine with the given \emph{inherited} attributes and returns the \emph{synthesized} attribute records. The syntax for invoking a visit is \textbf{invoke\_I\_v \emph{wrapper inhs}}. The \textbf{I} indicates the interface name, the \textbf{v} the visit name, \textbf{inhs} stands for the record containing the inherited attributes and finally \textbf{wrapper} is the wrapper function to call. For every \textbf{datasem} \rcore defines a wrapper \textbf{dnt\_I} and for every \textbf{sem} function the name which was explicitly given is used (more on this later).

The expression example can be scaled up by adding variables to the expressions. Two extra constructors need to be introduced to \emph{Expr}. They correspond to introduction and elimination:	

\begin{code}
  con Var
    nm      :: String
  con Let
    nm      :: String
    exp     :  Expr
    body    :  Expr
\end{code}

In order to be able to evaluate variables an \emph{environment} needs to be passed down through the tree to collect all the declarations. This is done by modifying the \emph{interface} of \emph{Expr}. The new \emph{interface} definition is:

\begin{code}
itf Expr
  visit eval 
    inh ast  :: Expr
    inh env  :: Env
    syn v    :: Int
    syn env  :: Env
\end{code}

A new attribute \emph{env} is added which is both a \emph{synthesized} and \emph{inherited} attribute. Strictly speaking the \emph{env} can only be an \emph{inherited} attribute, however both are needed to allow access to variables introduced in the left branch in Expr in the right branch.

Now that the type and interface have been extended the next step is to extend the datatype semantics to support the new clauses.

\begin{code}
   clause Var
     loc.val  = lookup loc.nm lhs.env
     lhs.v    = fromMaybe (error ...) loc.val
   clause Let
     loc.env   = (loc.nm, exp.v): lhs.env
     body.env  = loc.env
     lhs.v     = body.v
\end{code}

The code for |Var| and |Let| show that |lhs| is used both for synthesized and inherited attributes. Which is intended is determined by the way it is used (see section \ref{bindings}).

Sometimes special types such as Lists are needed. The next example shows how list support is provided in \rcore by adding the ability to evaluate a list of expressions.

\subsection{Special types}
At the time of writing \rcore only supports \emph{List}s in the category of special types, but can easily be extended to support any product datatype like |Maybe| and |Map|.

\subsubsection{Lists}
Lists are declared using the \textbf{type} keyword. The syntax is very familiar to Haskell programmer: \[ \textbf{type} \hspace{5pt} \emph{name} : [\emph{type}] \]

Declaring the list type \emph{Expr} would look like:

\begin{figure}[H]
\begin{code}
type Exprs : [Expr]
\end{code}
\caption{Exprs declaration using a list type}
\label{type:exprs}
\end{figure}

Figure \ref{type:exprs} declares the \emph{nonterminal} Exprs. This definition is semantically almost equivalent the data declaration in figure \ref{type:lists}. It introduces two type constructors \emph{Nil} and \emph{Cons} along with some extra attributes.

\begin{figure}[H]
\begin{code}
data Exprs
  con Nil
  con Cons
    hd  : Expr
    tl  : Exprs
\end{code}
\caption{Syntactically equivalent definition of the Exprs type}
\label{type:lists}
\end{figure}

Next to creating the syntactical information, ruler-core also generates some \emph{Interface} declarations for every list type. 

The interface declared for the \emph{Exprs} example in figure \ref{type:lists} is equivalent to:

\begin{figure}[h!]
\begin{code}
itf Exprs
  visit exprs_visit
    inh ast :: Exprs
\end{code}
\caption{Ruler-core interface declaration syntax}
\label{itf:exprs}
\end{figure}

A special \emph{inherited} attribute \textbf{ast} is declared on which \emph{matches} will be performed in clauses. This interface itself is not useful, instead the preprocessor enforces that at least one of the visits declared for the \emph{non-terminal} Exprs contain an inherited attribute \emph{ast}. If this is not the case an error will be generated.

With figure \ref{type:lists} and \ref{itf:exprs} in mind the real interface to our \emph{Exprs} type can be declared along with the corresponding \textbf{datasem}.

\begin{figure}
\begin{minipage}[t]{0.4\linewidth}
\begin{code}
itf Exprs
  visit eval 
    inh ast  :: Exprs
    inh env  :: Env
    syn v    :: [Int]
    syn env  :: Env
\end{code}
\end{minipage}
\begin{minipage}[t]{0.6\linewidth}
\begin{code}
datasem Exprs
   clause Cons
     hd.env   = lhs.env
     tl.env   = lhs.env
     lhs.v    = hd.v : tl.v
     lhs.env  = lhs.env
   clause Nil
     lhs.env  = lhs.env
     lhs.v    = []
\end{code}
\end{minipage}
\end{figure}

A lot of time is spent just copying over the \emph{env} attribute without actually doing anything with it. The larger the program gets the more problematic this becomes. For that \rcore has a special mechanism:

\subsection{Default rules}
Default rules are a way of specifying default values for attributes that will be used in case explicit values are not given for the attributes in question. This is particularly useful for when it is required to do nothing in some clauses.
When using a \emph{default} rule, all the visits of all the children of the current interface become active. This means every \emph{inherited} attribute must be filled in.
These rules are for both \emph{inherited} and \emph{synthesized} attributes. In case an attribute is both then the rule applies to both. 

The values are collected in such a way that they are added to the list in order of occurrence. Meaning the \emph{head} of the list contains the value at the root and the \emph{tail} of the list contains the last value visited. There are two kinds of \emph{default} rules:
\begin{description}
\item[\textbf{default}] { The \textbf{default} keyword gives an error when none of the children of an attribute for which the rule is defined for return a result. }
\item[\textbf{default?}] { The \textbf{default?} returns the empty list instead of an exception when no child has the \emph{synthesized} attribute in mention. For \emph{inherited} attributes, the initial value \emph{lhs.attribute} is also added to the list. }
\end{description}

\begin{figure}[h!]
\[
\textbf{default[\textit{?}]} \hspace{5pt} \emph{attribute} = \emph{function}
\]
\caption{Syntax for default expressions}
\label{default:syntax}
\end{figure}

The default rules look at the attribute names and not the types. If a child defines an attribute for which there is a \emph{default} rule defined but where the \emph{type} of this attribute is different then the type of the other element of the list, then a type error will be generated by the Haskell compiler since Haskell lists are heterogeneous.

Using default rules we can simplify the definition of the Exprs |datasem|:

\begin{code}
datasem Exprs
   default? env = last
   clause Cons
     lhs.v = hd.v : tl.v
   clause Nil
     lhs.v = []
\end{code}

\subsection{Multiple tree traversals}
The previous examples are all possible with uuagc and so with standard attribute grammar, whereas the following example is not. To show how to traverse multiple trees at the same time, and the higher-orderedness of \rcore the following example deals with how to compare two trees for equality.

The interesting points of this example are being able to traverse two AST at a time, and compare them while evaluating the tree at the same time.
If the trees turns out to be equal return, then return a new tree with the value at the given node.

The easiest one to start with is the \emph{Operator} terminal. An interface that allows the comparison of two operators needs to be defined:

\begin{figure}[H]
\begin{minipage}[t]{0.3\linewidth}
\begin{code}
data Operator
  con Plus
  con Minus
\end{code}
\end{minipage}
\begin{minipage}[t]{0.7\linewidth}
\begin{code}
itf OperatorEq
  visit compare
    inh op1  :: Operator
    inh op2  :: Operator
    syn eq   :: Maybe Operator
\end{code}
\end{minipage}
\caption{Interface to compare two operators}
\end{figure}

The interface \emph{OperatorEq} declares one visit "compare" and within this visit three attributes. The two inherited attributes are the inputs (the two operators to compare) and the synthesized attribute \emph{eq} is the result. The same result pattern will be followed in the rest of the example, if the inputs are equal we return one of the inputs, if not Nothing is returned.

Earlier in section \ref{semantics} it was explained that there are two types of semantic functions. \emph{Data type semantic} functions were explained in that section. This section however requires the second method of implementing semantic functions. The semantic function to compare two Operators is:

\begin{code}
{
eqOp = sem eqOp : OperatorEq monad IO
         visit compare
           default? eq = const False
           clause Plus
             match Operator.Plus@loc = lhs.op1
             match Operator.Plus@loc = lhs.op2
             
             lhs.eq = return Operator_Plus
           clause Minus
             match Operator.Minus@loc = lhs.op1
             match Operator.Minus@loc = lhs.op2

             lhs.eq = return Operator_Minus
           clause other
}
\end{code}

\subsubsection{Sem}
\begin{figure}[!h]
\begin{code}
<name> = sem <internal_name> : <Interface> [monad <type>]
          {visit <name>
             {clause <name>
                ...
             }
          }
\end{code}
\caption{Syntax definition of a sem function}
\label{sem:syntax}
\end{figure}

This method of declaring semantic functions allows the declaration of a semantic function for an arbitrary interface. The different components of figure \ref{sem:syntax} are decomposed as:

%\begin{figure}[ht!]
\begin{description}
\item[\textbf{\textit{name}}] This is the name of a semantic function. It is also the name of the Haskell function that will ultimately be generated. The same naming rules apply as for normal Haskell functions.
\item[\textbf{\textit{internal\_name}}] The internal name is only used internally and is not of real importance for anything done in this thesis.
\item[\textbf{\textit{Interface}}] Interface is the interface for which we are defining a semantic function. Every \emph{synthesized} attribute of the \emph{interface} must be assigned in the semantic function.
\item[\textbf{monad \textit{type}}] { This is optional. Failure is handled by backtracking. This value specifies the monad to be used while backtracking. This also forces the monad's type from a completely polymorphic type to a more concrete type. The \textit{type} specified needs to be an instance of \emph{Monad} and \emph{MonadError} due to backtracking of match failures being done by catching errors.}
\item[\textbf{visit \textit{name}}] The name of the visit for which a semantic function is being defined. At least one visit out of the interface should be implemented, but every \emph{synthesized} attributes must be filled in.
\item[\textbf{clause \textit{name}}] The clause \textit{name} needs to be a unique name as the names are used to differentiate the clauses from one another inside the semantic function.
\end{description}
%\end{figure}

A \textbf{sem} function does not enforce any kind of condition on visits specified. There is however one important layout rule: every \emph{visit} must be indented at least as deeply as preceding \textbf{sem} keyword. If this is not the case a parse error will be generated.

\subsection{Syntax}
\Rcore like its ancestor \emph{UUAGC} is implemented as a preprocessor for the language Haskell. From a .rul file pure Haskell code, which is callable from any other Haskell function is generated. How every bit of the ruler code is translated to Haskell is outside the scope of this thesis, but those interested can find it in detail in the paper by Arie Middelkoop\cite{visitag}.

Though it provides various syntactical elements, one of the most important to remember is that the right hand side of an equal sign ($=$) contains Haskell code, which means it is possible to call any Haskell function including library functions from the RHS. 

\subsubsection{Haskell}
In order to \emph{escape} into Haskell mode \emph{curly} braces \{ \} are used. The declaration of a module header and importing Haskell modules can be done with:

\begin{figure}[!h]
\begin{code}
{
{-# LANGUAGE BangPatterns #-}
module Eval where

import Control.Monad.Error
}
\end{code}
\caption{Example Haskell mode escape}
\end{figure}

The location and indentation of the braces do not matter, the code between the braces is copied verbatim to the generated Haskell file. For aesthetic reasons the \emph{curly} braces are usually placed at the beginning of the lines and on a line of their own. 

The next part will show how to compare two expressions and declare  explicit children in a clause. The place to start is by defining the interface:

\begin{code}
itf Compare
  visit compare
    inh exp1  :: Expr
    inh exp2  :: Expr
    inh env   :: Pair Env
    syn env   :: Pair Env
    syn exp   :: Maybe Expr
\end{code}
Comparing and evaluating two trees at once require two Environment. One per tree. The |Pair| type is just an alias for a 2-tuple of the same type. (e.g  (Env, Env))

\subsection{Children}
One of the benefits of \rcore over traditional AG systems such as \emph{UUAGC} is the ability to synthesize new trees during attribute evaluation. To accomplish this \rcore allows the declaration of new child \emph{non-terminals}. Due to space constrains only a piece of the semantics for the \emph{Compare} interface will be shown. To see the full source please consult the Appendix \ref{appendix:compare}.

\begin{code}
eq = sem eq : Compare monad IO
       visit compare
         default? env = last
         default? exp = const Nothing
         clause Num
           match Expr.Num@e1 = lhs.exp1
           match Expr.Num@e2 = lhs.exp2
           
           loc.eq  = e1.val == e2.val
           lhs.exp = guard loc.eq >> return (Expr_Num e1.val)
         clause Expr
           match Expr.Expr@l = lhs.exp1
           match Expr.Expr@r = lhs.exp2
           
           child left : Compare = eq
           left.exp1 = l.exp1
           left.exp2 = r.exp1
           
           child op : OperatorEq = eqOp
           op.op1 = l.op
           op.op2 = r.op
           
           child right : Compare = eq
           right.exp1 = l.exp2
           right.exp2 = r.exp2
           
           lhs.exp = liftM3 Expr_Expr left.exp op.eq right.exp
\end{code}

Contrary to a \textbf{datasem} there is no real distinction between terminals and nonterminals in a \textbf{sem}. There are no implicit children. If a routine is to be invoked it has to be explicitly declared.

\begin{figure}[h!]
\[
\textbf{child} \hspace{5pt} \emph{name} : Interface \hspace{5pt} [= sem\_name]
\]
\caption{child declaration syntax}
\end{figure}

This declares a new \textbf{child} \emph{name} belonging to the nonterminal (Interface) defined by the specified visit\cite{visitag}. By default the co-routine to execute is \emph{id} unless otherwise specified. When invoking a child of a nonterminal for which we defined a datasem the \emph{sem\_name} doesn't have to be defined.

\subsection{Ordering}
The order of statements are largely irrelevant, however \textbf{match} statements will always be executed before any other statement declared beneath them. They are also treated as assertions. Statements and assignments can be in any order, when they are scheduled they will be ordered based on their dependencies. Which is why

\begin{code}
l.e = loc.e
loc.e = m.l
\end{code}

is perfectly valid. Because of the dependence of \emph{l.e} on \emph{loc.e}, \emph{loc.e} will be floated above \emph{l.e}. This dependency resolving will also disallow any cycles inside the dependency graph. Cycles would cause an error to be generated by the preprocessor.
